\documentclass[a4paper, 11pt]{article}

% Layout
\usepackage[a4paper, left=3cm, right=3cm, top=2cm, bottom=3cm]{geometry} % kleinere Ränder
\usepackage{parskip}

% Umlaute in der Datei erlauben, auf Deutsch umstellen
\usepackage[T1]{fontenc}
\usepackage{lmodern}
\usepackage[utf8]{inputenc}
\usepackage[english, ngerman]{babel} % Abgaben auf DEUTSCH
%\usepackage[english]{babel} % for submissions in ENGLISH

% Mathesymbole und Ähnliches
\usepackage{amsmath}
\usepackage{mathtools}
\usepackage{amssymb}
\usepackage{microtype}
\usepackage{stmaryrd}

% Grafiken und PDFs einfügen
\usepackage{graphicx}
\usepackage{pdfpages}

% Abbildungen
\usepackage{tikz}
\usetikzlibrary{arrows, calc}

% Reelle, Natürliche, Ganze, Rationale Zahlen
\newcommand{\R}{\ensuremath{\mathbb{R}}}
\newcommand{\N}{\ensuremath{\mathbb{N}}}
\newcommand{\Z}{\ensuremath{\mathbb{Z}}}
\newcommand{\Q}{\ensuremath{\mathbb{Q}}}

% Fraktur für Strukturen
\newcommand{\A}{\ensuremath{\mathfrak A}}
\newcommand{\B}{\ensuremath{\mathfrak B}}
\newcommand{\I}{\ensuremath{\mathfrak I}}

% Makros für logische Operatoren
\newcommand{\xor}{\ensuremath{\oplus}} % exklusives oder
\newcommand{\impl}{\ensuremath{\rightarrow}} % logische Implikation

% Meistens ist \varphi schöner als \phi, genauso bei \theta
\renewcommand{\phi}{\varphi}
\renewcommand{\theta}{\vartheta}

% Aufzählungen anpassen (alternativ: \arabic, \alph)
\renewcommand{\labelenumi}{(\roman{enumi})}

\begin{document}

% OPTIONAL: Kopfzeile / header
MaLo~SoSe~2023
\hfill
{\Large
    Ãœbungsblatt 0 % <--- Blattnummer / sheet number
}
\hfill
\today

% Bitte prüfen Sie Ihre MOODLE-ABGABEGRUPPE, damit die Abgabe Ihnen richtig zugeordnet wird.
% Sie KÖNNEN hier Ihre Namen und Matrikelnummern für sich angeben, aber die Angaben werden IGNORIERT.

% Please check your MOODLE SUBMISSION GROUP in order to ensure that your submission is attributed to you.
% You MAY provide your names and matriculation numbers here for your own use, but they will be IGNORED.

\hrule

\section*{Aufgabe 1}

\LaTeX{} eignet sich gut zum Lösen der Aufgaben.
Im folgenden werden nützliche \LaTeX{}-Befehle für diese Veranstaltung vorgestellt.
Bitte achten Sie auch auf die Kommentare in der \texttt{.tex}-Datei.

Am Ende finden Sie Hinweise, wie andere Bilder und Dokumente hier eingebunden werden können.
So können Sie zum Beispiel handschriftliche Lösungen in Ihr Dokument einbinden
oder diese Vorlage nur als Titelseite verwenden und den Rest Ihrer Abgabe einbinden.


\section*{Aufgabe 2}

Formeln lassen sich im Text angeben, z.B.\ $\varphi = \varphi_1 \lor \varphi_2$,
oder abgesetzt
\[
    \varphi = \varphi_1 \lor \varphi_2.
\]


\section*{Aufgabe 3}

\paragraph*{a)}

Für Teilaufgaben kann man Aufzählungen verwenden:
\begin{itemize}
\item als Aufzählung
\item ohne Nummerierung,
\end{itemize}

\paragraph*{b)}

oder auch
\begin{enumerate}
\item mit
\item Nummerierung.
\end{enumerate}


\section*{Aufgabe 4}

Hier ein paar nützliche Symbole:
\begin{itemize}
\item $\lor$, $\land$, $\neg$
\item $\impl$, $\xor$
\item $\forall$, $\exists$
\item $\N$, $\R$, $\Q$
\item $\A$, $\B$, $\I$
\item $\llbracket \varphi \rrbracket$
\end{itemize}


\section*{Aufgabe 5}

Kleine Diagramme lassen sich ebenfalls erstellen, zum Beispiel:

\begin{center}
\begin{tikzpicture}
\node (1) {$v_1$};
\node [right of=1] (2) {$v_2$};
\node [below of=2] (3) {$v_3$};
\draw (1) -- (2);
\draw (1) -- (3);
\end{tikzpicture}
\end{center}

oder auch etwas komplizierter:

\begin{center}
% Stil festlegen, der dann für alle Knoten verwendet wird
\tikzset{dot/.style={circle, draw=black, fill=black, inner sep=0pt, minimum size=5pt}}
% das eigentlich Diagramm
\begin{tikzpicture}[node distance=1.5cm]
\node [dot,label={left:$v$}] (1) {};
\node [dot,right of=1] (2) {};
\node [dot,below of=2,label={right:$w$}] (3) {};
\node [dot,below of=1] (4) {};
\path [draw=black, ->, >=stealth', shorten <=2pt, shorten >=2pt] % Optionen nur für das Aussehen
    (1) edge (3)
    (1) edge (4)
    (2) edge (3)
    (4) edge [loop left] ()
    (1) edge node [above] {$e$} (2);
\end{tikzpicture}
\end{center}

Alternativ können auch externe Dateien eingebunden werden (z.B.\ Bildformate, PDF).
Dazu nutzen Sie den Befehl \texttt{\textbackslash includegraphics\{dateiname\}}.

%\begin{center}
%\includegraphics{dateiname}
%\end{center}


\section*{Aufgabe 6}

Falls Sie die Aufgaben anders gemacht haben
und bereits eine PDF-Datei \texttt{beispiel.pdf}
oder Bilddateien \texttt{bild1.png}, \texttt{bild2.jpg}, $\ldots$
haben, können diese hier leicht eingebunden werden.

\begin{itemize}
\item Ganze Seiten aus PDF-Dateien fügen Sie mit
\[
\texttt{\textbackslash includepdf[pages=-]\{beispiel.pdf\}}
\]
ein. Dabei können Sie mit \texttt{pages} alle Seiten
oder einen Bereich auswählen.
Für Dateien im Querformat nutzen Sie \texttt{landscape=true}.
\[
\texttt{\textbackslash includepdf[pages=-, landscape=true]\{beispiel.pdf\}}
\]
\emph{Bitte vermeiden Sie wechselnde Seitenformate in Ihrer Abgabe, wenn möglich.}

\item Fotos oder Scans von Seiten und/oder Aufgaben können Sie mit
\[
\texttt{\textbackslash includegraphics[width=\textbackslash linewidth]\{dateiname\}}
\]
einfügen. Die Skalierung können Sie beispielsweise mit \texttt{width} einstellen.
Bitte achten Sie unbedingt darauf, dass alles \emph{gut lesbar} ist
und machen Sie bei schlechter Qualität lieber zwei oder mehr Bilder pro Seite.
Mit \texttt{\textbackslash clearpage} beginnen Sie eine neue Seite.
Das Bild kann gegen den Uhrzeigersinn mit \texttt{angle=90} oder
im Uhrzeigersinn mit \texttt{angle=270} gedreht werden.
Wählen Sie die Skalierung nach Ihrem Ermessen, sodass alles passt,
zum Beispiel macht \texttt{width=0.9\textbackslash linewidth} das Bild etwas kleiner.

\emph{Bitte sorgen Sie dafür, dass alle Bilder und Scans richtig gedreht, vollständig sichtbar und gut lesbar sind.}
\end{itemize}

%\includepdf[pages=-]{beispiel.pdf}
%\includepdf[pages=-, landscape=true]{beispiel.pdf}

%\clearpage

% Hier ist ein Bild auf einer Seite (im Uhrzeigersinn gedreht und vergrößert).

%\includegraphics[width=1.3\linewidth, angle=270]{bild_ganze_seite}

%\clearpage

% Hier sind zwei Bilder auf einer Seite.
% Bei schlechter Bildqualität hilft es, von jeder Seite 2 oder mehr Fotos oder Scans zu machen.

%\includegraphics[width=0.95\linewidth]{bild_halbe_seite_1}

%\includegraphics[width=0.95\linewidth]{bild_halbe_seite_2}

\end{document}